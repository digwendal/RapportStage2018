%%
%% Author: gdidot
%% 7/13/18
%%

% Preamble
\documentclass[11pt, a4paper, pdftex]{article}
\setlength{\headheight}{33pt}

% Packages
\usepackage[a4paper,top=3cm,bottom=2cm,left=3cm,right=3cm,marginparwidth=1.75cm]{geometry}
\usepackage{a4wide}
\usepackage[french]{babel}
\usepackage[utf8]{inputenc}
\usepackage{graphicx}
\usepackage{fancyhdr}
\usepackage{lastpage}
\usepackage[T1]{fontenc}

% Document
\begin{document}
    %Page de garde

    %Info pour maketitle
    \title{Rapport de Stage: \\ Portage de Malai pour l'environement WEB}
    \author{Gwendal \bsc{Didot}
    \\
    \multicolumn{1}{p{.7\textwidth}}{\centering \vspace{0.25cm} Master 1 Informatique spécialisation Sécurité, Système et Réseaux à l'ISTIC, Université Rennes 1}
    \\
    \multicolumn{1}{p{.7\textwidth}}{\centering \vspace{0.25cm} Encadré par Arnaud \bsc{Blouin}}}

    %Setup de la page de garde
    \maketitle
    \begin{center}
        Stage réalisé du \date{15 mai 2018} au \date{31 août 2018} \\ dans l'équipe DiverSE \\ des laboratoires INRIA/IRISA \\ sur le campus de Beaulieu, Rennes
    \end{center}
    \thispagestyle{empty}
    \newpage

    \tableofcontents

    \fancyhf{}
    \pagestyle{fancy}
    \fancyhead[L]{\includegraphics[height=1.0cm]{../assets/logo_istic.png}}
    \fancyhead[C]{\leftmark}
    \fancyhead[R]{\includegraphics[height=1.0cm]{../assets/logo_univ_rennes1.png}}
    \fancyfoot[C]{\thepage/\pageref{LastPage}}
    \newpage
    \section{Présentation de l'entreprise}\label{sec:presentr}
    \vspace{1cm}
        \subsection{INRIA}\label{subsec:inria}
            \paragraph{}
                L'Institut Nationale de Recherche en Informatique et en Automatique (INRIA) est un établissement de recherche publique en science du numérique,
                sous la tutelle des ministères de l'Enseignement supérieur et de la Recherche et de l'Économie et des Finances.
                Il est composé de multiples organisation de recherche, et est séparé en différents centres de recherches autonomes à travers la France.
    \vspace{1cm}
    \subsection{IRISA}\label{subsec:irisa}
            \paragraph{}
                L'Institut de Recherche en Informatique et Systèmes Aléatoires (IRISA) est un laboratoire de recherche en informatique, automatique,
                traitement du signal et des images et en robotique.
                L'IRISA est un laboratoire très présent en Bretagne, et est sous la tutelles de plusieurs organismes, dont l'INRIA\@.
    \vspace{1cm}
    \subsection{DiverSE}\label{subsec:diverse}
             \paragraph{}
                L'équipe de recherche DiverSE est membre de l'IRISA et de l'INRIA. Son domaine de recherche se porte sur l'ingénierie logiciel, où elle développe de nouveaux modèles,
                méthodes et théories permettant de répondre aux besoins causés par la diversification des façons de penser, déployer et faire évoluer les systèmes fortement axés sur les logiciels.
                Cette équipe a pour avantage d'avoir des liens étroits avec l'industrie, ce qui se retrouves dans la façon de travailler de ces membres.
    \newpage
    \section{Objectif du stage}\label{sec:objsta}
    \vspace{1cm}
        \subsection{Présentation de Malai}\label{subsec:premal}
            \paragraph{}
                Les interfaces utilisateurs sont un composant essentiel pour un logiciel, mais actuellement les moyens donnés au développeur pour réaliser
                une interface sont assez rudimentaires, demandant un développement très bas niveau, et n'offre pas de moyen de facilement réutiliser du code déjà fait.
                Pour palier à ce problème, Malai à été créé pour permettre au développeur d'interface utilisateur de pouvoir le faire bien plus facilement, et avec un gain de temps non négligeable.
    \vspace{1cm}
        \subsection{Objectif du stage}\label{subsec:objsta}
            \paragraph{}
                Ayant vocation à être utiliser par tous développeurs d'interface utilisateur, Malai se doit d'être présent pour l'environement WEB, du simple site html/javascript au frameworks les plus complexes.
                Mais n'était pour le moment développer que pour l'environement applicatif, avec JavaFX. Le but du stage était donc de porter le code de Malai en Java
                vers Typescript, ainsi que réaliser des applications permettant de montrer les avantages de MalaiTS (la version WEB de Malai).
    \newpage
    \section{Travail réalisé}\label{sec:trarea}
    \newpage
    \section{Bilan du stage}\label{sec:bilsta}

\end{document}
